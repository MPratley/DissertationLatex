\documentclass[10.5pt,twoside,openright]{report} 
\usepackage[utf8]{inputenc}
\usepackage[english]{babel}
% \usepackage{verbatim}
% \newcommand\wordcount{\verbatiminput{words.sum}}

% \setlength{\parskip}{\baselineskip}%
\setlength{\parskip}{2mm}%
\setlength{\parindent}{0mm}%

\title{Investigating the effect of measurement error on predicting future brain
activity with a computational model}
\author{Marcus Pratley}
\date{2020}

% % Run texcount on Tex-file and write results to a sum-file
% \immediate\write18{texcount -1 -sum -merge Chapter*.tex -out=words.sum}

\usepackage[ungrad, hyperref]{edmaths}
\usepackage{epigraph}
\usepackage[autostyle]{csquotes}
\usepackage[style=apa]{biblatex} 
\addbibresource{refsv2.bib}

\usepackage[default,light,bold]{sourceserifpro}
\usepackage[T1]{fontenc}

\usepackage{mathtools}
\usepackage{multirow}
\usepackage{amsmath, amsthm, amssymb, amsfonts}
\usepackage{breakcites}
\usepackage{subcaption}
\usepackage{graphicx}
\usepackage{placeins}
\usepackage{sidecap}
\usepackage{mhchem}
% \usepackage{setspace}
% \doublespacing

\usepackage{environ}
\NewEnviron{myequation}{%
\begin{equation}
\scalebox{1.5}{$\BODY$}
\end{equation}
}


% \usepackage{tocloft}
% \usepackage[small]{caption}

\captionsetup[figure]{font=it} 

% some auxiliary lengths for aligning the captions
\newlength\mylena
\newlength\mylenb 

% syntax: \MyCaption{First numbered caption}{Second unnumbered caption}
\newcommand\DoubleCaption[2]{%
  \captionsetup{belowskip=-\baselineskip}
  \settowidth\mylena{\small\itshape\figurename~\thefigure. #1.}
  \settowidth\mylenb{\small\itshape #2.}
  \caption{#1} 
  \caption*{\hspace*{\dimexpr\mylenb-\mylena\relax} (#2)}
  \setlength\belowcaptionskip{\baselineskip}
}

% % settings for the list of figures
% \renewcommand\cftfigpresnum{Figure }
% \renewcommand\cftfigaftersnum{.}
% \newlength\mylend
% \settowidth\mylend{\cftfigpresnum\cftfigaftersnum}
% \addtolength\cftfignumwidth{\mylend}

% \usepackage[bitstream-charter]{mathdesign}
% \usepackage[T1]{fontenc}

\begin{document}

\csname @openrightfalse\endcsname

\maketitle

\declaration

\dedication{With thanks to Frances Hutchins and Professor Marcus Kaiser, who have both been a great help throughout the unusually fraught environment of creating this dissertation. \\ \vspace{1ex} Extra thanks also to my family, and parents in particular, for all their love and support over the course of my degree. }

\begin{abstract}
This paper will discuss the plausibility of accurate computational brain
simulation, given the current limitations of creating digital copies of the
human mind. A spiking neural network has been programmed in the Python programming language, and is used to evaluate the implication of error introduced to a network during simulation. 

My analysis shows the expected correlation between error introduced to a
simulation and the reduction in simulation accuracy, in addition, that the
presence of synaptic plasticity in the network has a mitigating effect on this
reduction in accuracy, increasing the network resilience.

% Modern neural imaging introduces error in numerous ways, TODO
\end{abstract}

\def\table{\def\figurename{Table}\figure}
\let\endtable\endfigure
\renewcommand\listfigurename{List of Figures and Tables}

\tableofcontents
\listoffigures
\csname @openrighttrue\endcsname

\chapter{Introduction}

The topic of creating artificial life has been subject to fervent debate for
hundreds of years, but at the advent of modern computing this conversation took
a turn towards the technological. In his 2005 book “The Singularity Is Near:
When Humans Transcend Biology” author Ray Kurzweil predicts not that we will
create a new species, but that “The Singularity will allow us to transcend [the]
limitations of our biological bodies and brains”
\parencite{kurzweil_singularity_2006}.
In short, we will be able to transfer our
consciousness to simulations of the human brain. The implication of this is
that, assuming inexhaustive power and processing to maintain the simulation,
humanity would be made immortal. However, there are a number of constraints
technologically that must be reckoned with before this could even be attempted
\parencite{bostrom_whole_2008}, alongside the moral quandaries that are ever
present when discussing the evolution and modification of what it means to be
human. 

When discussing brain emulation, we envisage the “uploading” of
consciousness, which in basic terms is the measurement of electrical activity
and neural connections of a brain, and in theory would retain every mental
process, keeping them intact. Using current known techniques, this
measurement process would introduce error in numerous ways; It is perhaps
possible that the resolution of the scanning equipment is not large enough to
correctly capture the distances between neurons, or that changes in the network
will occur while measurements are taken. It is therefore unclear what impact
this measurement error would have on a simulation of the human brain.

\section{Motivation}

This project aims to investigate the effect of measurement error on the accuracy
of a brain simulation with and without synaptic placticity, and to assess the
feasibility of predicting future brain activity, given the performance of
current imaging techniques.

The primary outcome of this project will be to provide Python tooling that is
capable of running simulations in a parallel manner across multiple threads,
that can be modified during runtime to model the effect of measurement error
when“uploading”a brain into a simulation. 

\section{Aims and Objectives}

\begin{itemize}
    \item \textbf{Analyse and compare three published neural simulation packages.}
          Developing an understanding of simulation packages and the models that
          they implement will help me inform my design decisions for my own
          implementation of this feature. This objective will have been
          completed when I have summarised and understood the aims and features
          of three existing packages.
    \item \textbf{Identify the experiments that should be performed to determine
    the relation between measurement error in data from imaging a brain, and the
    performance of a neural simulation of such data.} In order to prove that
    that measurement error has an effect on predicting future brain activity, I
    will define experiments that can be performed within the limits of this
    project. Each experiment must contribute either to the argument that
    measurement error has a major negative impact on the accuracy of a neural
    simulation, or that introducing plasticity into the simulation minimises
    this effect over time.
    \item \textbf{Identify the requirements and features that my simulation tooling should implement to be capable of performing the project experiments.}
    % \item \textbf{Identify the requirements of a routine within VERTEX that is
    %           capable of saving a snapshot of a running simulation.} I will need
    %       to gather data relating to the functionality and design of a
    %       simulation snapshot saving system. This will include the API
    %       presented to the user of the software, required data structures to
    %       store information, and choosing functionality which forms the
    %       minimal viable product. To achieve this, I must have collected
    %       enough research to make conclusive design and functionality
    %       decisions.
    \item \textbf{Identify the parameters in a brain simulation that can be
              modified during the simulation to emulate the measurement error of
              uploading the human brain.} Many of the parameters required in the
          creation of a simulation, such as the number of layers or the
          positions and links between neurons could be modified during a
          simulation using the tooling this project aims to develop. This
          objective will be complete when I have identified these
          parameters, and justified why they have been chosen over others.
    \item \textbf{Analyse the performance of simulations that diverge with
              different measurement errors from a starting simulation.} This
          final objective will assess whether the software designed and
          developed during the project can correctly demonstrate the effect
          of measurement error on brain simulation. 
\end{itemize}

\section{Experiments and Hypotheses}



\section{Document Structure}

\subsection*{Background and Literature Review}
This chapter aims to convey the general state of what is considered mainstraim
academic and industry opinion, and some of the more cutting edge approaches in
each of the domains that this project touches on.

\subsection*{Planning and Development}
This chapter will discuss the methodology by which project deliverables have
been developed, and explain the rationale of certain design decisions. In
particular, functionality of the neural network simulation will be
discussed, as will the simulation parameters.

\subsection*{Experiments and Results}
The rudimentary correctness of the Python simulation is verified here, and each
of the experiments that form the basis of the arguments made in this
dissertation are expanded on and their results are explained. 

\subsection*{Analysis and Evaluation}
An analysis and evaluation will review the results in the previous chapter and
show how the results reflect on the overall project aims. The results of all the experiments are reviewed holistically and compared with
expected findings. 
\subsection*{Conclusion}
A summary of the project, and whether the aim and objectives have been met. The
overall sucess of the project is evaluated and future work that could be
completed in another project expanded on.
\chapter{Background and Literature Review}

\section{Imaging and Image Processing}

\begin{itemize}
    \item Prediction of future brain activity through simulation requires an accurate and detailed connectome of a brain, with synapses and neurons correctly located in space.\autocite{bostrom_whole_2008}
    \item The accuracy of such a model depends on the resolution of the imaging method used to create it. The error resulting from such a imaging method is the \textbf{measurement error}. 
    \item Depending on imaging procedure, brain matter may shift in composition during the course of the scan, which is the cause of \textbf{measurement drift}, itself a form of measurement error.
\end{itemize}

\subsection{Current Methods of Scanning the brain}

\subsubsection[Error induced through the imaging process]{Examination of Error induced through the imaging process}

\setlength{\tabcolsep}{4ex}
\renewcommand{\arraystretch}{1.1}
\begin{table}[ht]
    \centering
    \begin{tabular}{@{}llll@{}}
        Method              & Resolution                 & Time    & Error(approx.) \\
        \hline
        MRI                 & 6$\mu m$                   & 30mins  & 95\%           \\
        MRI microscopy      & 3$\mu m$                   & -       & 85\%           \\
        XRay microscopy     & 30nm                       & -       & 30\%           \\
        Electron microscopy & \textasciitilde 30nm-0.1nm & >3mnths & <1\%           \\
        Theoretical Ideal   & <5nm                       & <500s   & <1\%           \\
        \hline
    \end{tabular}
    \DoubleCaption{Comparison of imaging methods.}{Error approximated from size of dendritic spines.}
    \label{imagemethodcomparison1}
\end{table}
\setlength{\tabcolsep}{1ex}

\subsubsection[Error induced through low resolution]{Effect of Low Resolution} 
(this is easy just print something out
and scan it badly. Explain what kind of error in incurred.)

\section{Existing simulation software}

\subsection{VERTEX}
\autocite{tomsett_virtual_2015} \autocite{thornton_virtual_2019}
\subsection{LFPy}
\autocite{hagen_lfpy_2019} \autocite{hagen_hybrid_2016}
\subsection{BRIAN}
NEED A REFERENCE FOR BRIAN

\section{Concurrency in Simulation}


\chapter{Design, Development and Methodology}

In order to better understand the fundamentals of computational neuroscience, I
am taking a "from scratch" approach to building this project. While this
increases the time required to run simulation experiments, this approach allows
me to build confidence in my understanding of the maths and models that are
commonly used in the wider computational neuroscience community.

\section{Simulation Requirements}

\begin{itemize}
    \item \textbf{Spiking neural network is the target emulation layer}
    Justification should cite imaging requirements and state that this is an
    appropriate level of abstraction to answer the question of this
    dissertation. Refer to the table of emulation layers in lit review.
    \item \textbf{Requirement 2.} Reasoning for requirement two.
\end{itemize}

\section{Modelling mathematics}
"In order to program this thing I must first enumerate the differential equations
that describe the functioning of Neurons in this spiking neural network."
\subsection{Leaky Integrate and Fire}

TEMP:
\begin{figure*}[h]
    \centering
    \begin{equation}\label{eq:LIF_TC}
        T_C
    \end{equation}
    \begin{equation}\label{eq:LIF_RC}
        \frac{d V}{d t} = -\frac{V}{\tau_L} + \frac{I(t)}{C},
    \end{equation}
    \begin{equation}\label{eq:integ_LIF_RC_VL}
        V(t)= \frac{1}{C} \int_{0}^{t} e^{-\frac{(t-s)}{\tau_L}} I(s) ds
    \end{equation}
    $C$ is the membrane capacity, $g_L$ is the leak conductance and $V_L$ is the leak reversal potential.
    % \caption{Formulae for a Leaky Integrate and Fire Neuron (pre-threshold)}
    \label{LIFequation}
\end{figure*}

\subsection{Spike refractory periods}

\subsection{STDP}

\section{Transferring models to Python}

In order to flexibly mix
object oriented and function programming paradigms, I've chosen to develop these
models in the Python programming language. Python also has a healthy and active
ecosystem of libraries and development practices for scientific and statistical research.

Of particular note are the scipy libraries, \ldots

\subsection{Object oriented neurons}

\subsection{Synapses}

\subsection{Stochastic generation of networks}

\subsubsection{Grouping neurons}

\subsubsection{Function parameters instead of static parameters}

\subsection{Measuring spiking patterns across the network}

-Mark certain neurons as recording
-This can be done as a boolean flag on the neuron to tell it to keep a time
trace of its potential, or by inserting measurement objects into the simulation.
These measurement objects can take the form of 

\subsection{Comparing spiking patterns between two networks.}

Treating spikes as a binary on-off at a given point in time, it is possible to
calculate the hamming distance between two network spiking patters. 


% \chapter{Experiments and Results}
\chapter{Analysis and Evaluation}

\section{What do the results mean?}

\section{Are they successful? Are they expected?}

\section{Reflection on development and development methodology}

 - Reactive approach, the requirements of the software adapting as the results
 of experiments take shape.

 - This is substantially different my original plan of identifying hard
 requirements of the system and implementing. 

 - It's much easier to let loose requirements set the direction of travel and
 let experiments along the way determine the end result of the software.

 - This is partially due to an initial lack of knowledge around the problem
 domain. As I understood more about computational neuroscience, domains that
 seemed simple took on additional complexity, while previously intractable
 concepts became apparent.

- Another benefit of taking an experimental and incremental approach to the
development of the project is that tasks naturally break down into small and
meaningful chunks; each new piece of code does the correct amount of work to
solve a set purpose.

- The caveat to this approach is that it does not lend itself to a holistic
system design or consistent architecture. When defining and writing new
data experiments, the flexibility of decoupled components is ideal, but new code
tends to be purpose built and highly coupled. This would sometimes mean
re-writing code several times as previously finished code had new requirements. 

- In general, Conclusions, I’ve found that getting development methodology
“right” is both hard and sort of a misnomer. It depends on the product, the
team, their locations and the workload: what “right” means will change as your
product changes and you move from an experimental development cycle to one with
well formed requirements. One solid way of working throughout a project will
cause friction later down the line. 

A lot of time an effort both inside and out of the software development industry
is spent defining `Agile ways of working' \autocite{spolsky_you_2006}, however, from my experiences on this
project, it is better to be flexible and reject workflow rigidity. 

Flexibility in ways of working does have its drawbacks however; deadlines must
be set and adhered to throughout a project lest one finds themselves overworked
and unable to produce outcomes that matter externally. In this aspect, my
project would have benefited greatly from a larger sense of responsibility to
myself and the targets I set myself at the beginning of the year. By way of
example, I had originally planned that certain stages of my project could
overrun safely, while others must be completed on time. Sticking to this plan
was far harder than I originally anticipated, and in hindsight it would have been
wise to re-scope the project as soon as it was clearly the best course of action,
instead of carrying a metaphorical weight that it was clear couldn't reach the
finish line.

\section{Personal reflection of result success/lack thereof}

\section{Personal reflection of whole project experience}
 - Certain concepts were intractable for quite some time 
 - concepts that I considered understood made little sense when the time came to
 develop on them 
\chapter{Conclusion}

\section{What Have I learned}

% what we know of its function is experimentally determined, not calculated from
% the sequence. It would be wonderful to be able to take a sequence, plug it into
% a computer, and have it spit back a quantitative assessment of all of its
% interactions with other proteins, but we can't do that, and even if we could, it
% wouldn't answer all the questions we'd have about its function, because we'd
% also need to know the state of all of the proteins in the cell, and the state of
% all of the proteins in adjacent cells, and the state of global and local
% signaling proteins in the environment.
% https://scienceblogs.com/pharyngula/2010/08/17/ray-kurzweil-does-not-understa

\section{What went well?}

\section{What could have been improved?}

\section{Have I met all my objectives?}

End with a cool quote

\printbibliography[heading=bibintoc]

\end{document}
