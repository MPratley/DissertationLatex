\chapter{Conclusion}

\section{Reflection on personal objectives}

Over the course of researching, developing and writing this dissertation, I've
encountered many ideas, schools of thought, and applications of maths and
computer science that were abstract or unknown to me previously. In choosing a
dissertation that is rooted in a topic that is conceptually new to me, my
primary aim was to develop an original understanding of the limitations that
current imaging technology poses to the feasibility of emulating the whole human
brain. 

Throughout the process of creating this dissertation, I have refined my own
researching skills, and am now comfortable synthesising many sources
covering a topic to improve my own understanding, something that I did not have
much practice of beforehand. 

\section{Result reflections and future work}

Would have been good to investigate the effect of .

Future work should investigate the effect of different Hebbian-Learning models
when applying error to a simulated model, and whether inhibitory synapses have a
similar compensatory effect as excitatory synapses have been shown to in this dissertation.

It seems likely that synaptic plasticity is the key to running simulations of
imperfectly created models that contain error, particularly in synapses where
the precise details of placement and physical composition must be resolved at
the scale of nanometres when imaging. If this is the case, then any future
technology that aims to emulate, and even predict, brain activity must ensure
that its underlying simulation is capable of capturing the nuances of synaptic plasticity.

\section{Personal reflection of whole project experience}
 - Certain concepts were intractable for quite some time 
 - concepts that I considered understood made little sense when the time came to
 develop on them 


% what we know of its function is experimentally determined, not calculated from
% the sequence. It would be wonderful to be able to take a sequence, plug it into
% a computer, and have it spit back a quantitative assessment of all of its
% interactions with other proteins, but we can't do that, and even if we could, it
% wouldn't answer all the questions we'd have about its function, because we'd
% also need to know the state of all of the proteins in the cell, and the state of
% all of the proteins in adjacent cells, and the state of global and local
% signaling proteins in the environment.
% https://scienceblogs.com/pharyngula/2010/08/17/ray-kurzweil-does-not-understa

\subsection{Success in the project}

\subsection{Improvements in the project}

While further work needs to be taken to understand how larger networks
might behave with this imaging error, I believe that the work described in this
document is both original and  

\section{Have I met all my objectives?}

End with a cool quote

% Singler says that existential despair or existential joy will be the driving force behind the
% design and production of transhumanist technology.
% \autocite{singler_existential_2019}