\section{Spiking neural networks}

\begin{quote}
    `The brain computes! This is accepted as a truism by the majority
    of neuroscientists engaged in discovering the principles employed in the
    design and operation of nervous systems. What is meant here is that any
    brain takes the incoming sensory data, encodes them into various biophysical
    variables, such as the membrane potential or neuronal firing rates, and
    subsequently performs a very large number of illspecified operations,
    frequently termed computations, on these variables to extract relevant
    features from the input.'
    \begin{flushright}
        \textit{-\autocite{koch_biophysics_2004}}
    \end{flushright}
\end{quote}

The mammalian nervous system is principally composed of neurons and glial cells.
Glial cells primarily perform supplementary roles in the function and processing
of data in the nervous system, but it is still not fully understood what further
role they play \autocite{walz_role_1989}. Neurons are less numerous in the
nervous system, and are chemically unique in their functioning in the body.

\begin{figure}[t]
    \centering
    \includegraphics{figures/graphs/huxhog_spike.png}
    \caption[The typical form of a neuronal action potential]{The typical form
    of a neuronal action potential, where the potential difference between the
    resting and peak potentials is approx. $40-100 mV$.
        \\\small{Created by
    Nir.nossenson@wikipedia.com, licenced 
            \href{https://creativecommons.org/licenses/by-sa/4.0/deed.en}{CC BY-SA 4.0}}}
    \label{neuronalactionpotentialexample}
\end{figure}
\vspace{1ex}

Each neuron acts as a gate of sorts, holding or releasing its potential in
response to signals as part of a greater network of neurons. This release often
takes the form of a spike, as seen in figure \ref{neuronalactionpotentialexample}, depicting a typical neuronal action potential of a neuron in the nervous system.
Together, the neurons in these networks perform the signal processing and
routing that "computes" the many sensory inputs to the nervous system
\autocite{koch_biophysics_2004}. While not all neurons are spiking neurons, it %TODO percentage of neurons that are spiking?
is the properties of spiking neurons and the interactions between them that are
the focus of this review and ultimately this dissertation.

\subsection{Hodgkin–Huxley model}

In 1952, Alan Hodgkin and Andrew Huxley described a model of the squid giant
axon that accurately represents both the chemical changes and the form of the
action potential spike in the axon. \autocite{hodgkin_quantitative_1952}

The strength of the Hodgkin–Huxley model is in its adaptability. Every change in
a biological neuron during a spike, be it chemical or physical, can be mapped
across to changes in the Hodgkin–Huxley model, and more complex behaviour
can be recreated in the model thanks to its basis in the core chemistry of a
neuron. However, this complexity is not without pitfalls as the computational
power required to simulate such advanced models is so large,  that it is no
longer feasible to simulate it within a network.

\subsection{Faster neuron models for large scale computation}

While it can be desirable to accurately model and simulate the chemical
interactions between individual components of the nervous system, it is
computationally infeasible to do so on a scale similar to that of a real-world
system due to the computational complexity of the Hodgkin-Huxley equations.
Fortunately, as a network becomes larger, the exact mechanisms of its individual
components can be reliably approximated, and faster activation or threshold functions
that perform similarly to real ones can be substituted.

One such example of a simple model of a spiking neuron is the Leaky Integrate
and Fire (LIF) neuron. It uses a simple threshold function to approximate the
action potential spike generation of a Hodgkin–Huxley neuron
\autocite{trappenberg_fundamentals_2009}. As this is a relatively simple
approximation, it is tempting to assume that such a model would prove inaccurate
for anything more than basic prototypes, however LIF models have been shown to
replicate spiking patterns of more complex models with a low margin of error,
provided the models are under natural conditions
\autocite{teeter_generalized_2018}. This dissertation will go into more detail on
the equations and implementation of LIF neurons in section
\ref{Modellingmathematics} of this document.

\subsection{Hebbian learning and synaptic plasticity}

In biological networks, neurons adapt their relationships over time in reaction
to new environments or long term changes, in a process called "Hebbian learning"
\autocite{trappenberg_fundamentals_2009}. This process can be replicated in a
spiking neural network through adding a vector of weights to each neuron that
determines the relative connection strength of each pre synaptic input. These
weights will change over time according to the observed correlation between
pre-synaptic and post-synaptic spikes, forming Spike-Time Dependent Plasticity
(STDP) \autocite{iakymchuk_simplified_2015}. The concept and implementation of STDP is expanded on in chapter 3, starting with
equation \ref{eq:stdpdw}.
