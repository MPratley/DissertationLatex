\section{Computational models for brain simulation}

Summary of what it means to simulate a brain and common assumptions made when
doing so.

\subsection{Available neural simulation and modelling software packages}
\subsubsection{VERTEX}
VERTEX is a BLAH written in the MATLAB programming language that is designed to
X and Y. It is based on the premise that Z can be assumed. It contains
functionality to do THINGS and OTHER THINGS. 

In 20xx, VERTEX was updated to provide the option of modelling brain simulation
techniques such as paired pulse depression and long term potentiation. In short,
this involves Y CHANGES and allows the researchers using it to do KJH.

% BAD AND OLD SENTENCE This was
% achieved by making changes to the synapse model in VERTEX so that weights
% between connections could be updated over time (plasticity).

\autocite{tomsett_virtual_2015} \autocite{thornton_virtual_2019}
\subsubsection{LFPy}
LFPy is described by NAME to be a THING that does X well. It is based on the
premise that Y can be assumed. It is based on top of the THINGIE platform, which
is used by several labs around that world.
\autocite{hagen_lfpy_2019} \autocite{hagen_hybrid_2016}
\subsubsection{BRIAN}
NEED A REFERENCE FOR BRIAN
\subsection{Turning image data into a connectome}

\section{Similarities in existing software packages}

Each of the mentioned software packages mentioned above is capable of doing XYZ,
but each has a distinct end goal. The VERTEX authors state that "QUOTE", while
BRIAN is designed less for a specific research purpose and more to be a
generally adoptable library for other developers and researchers to use in their
own projects. The LFPy authors have taken a combined approach, with thorough
documentation of the project, but the tool is still understandably designed for
the research purposes of its authors.