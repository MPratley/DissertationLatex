\section{Simulation software requirements}
\label{SimulationRequirements}
% Justification should cite imaging requirements and state that this is an
% appropriate level of abstraction to answer the question of this
% dissertation. Refer to the table of emulation layers in lit review.

While there are many different levels and methods of emulation possible in the
field of computational neuroscience, for the purposes of this dissertation, a
spiking neural network is an appropriate level of simulation, such that network
changes that mimic measurement error in the brain scanning process can be
inserted into the simulation. It is, however, not necessary to precisely emulate
the chemical processes that take place within the biological nervous system, and
these processes may be mathematically approximated.

The following are the requirements that should be met by simulation software
that is suitable for research into the effect of measurement error on brain
simulation. I have identified these requirements on the basis of my research in the prior chapter.

\begin{itemize}
    \item \textbf{Use a clock-driven, discretization based approach for simulation.} The relative time of
          events to each other during simulation is required for comparison of
          potential distributions and implementing STDP. A real-time event based
          model would make comparisons between simulations difficult as processes in
          the background may take up computational resources.

    \item \textbf{Support the synaptic plasticity of weights between neurons.}
          The ability of a biological network to adapt over time is an important
          feature that facilitates network resilience and self-repair when damaged
          \autocite{trappenberg_fundamentals_2009}, and a networks' ability to learn
          and remember \autocite{eichenbaum_cognitive_2011}.
          % [TEMP] Justification should cite imaging requirements and state that
          % this is an
          % appropriate level of abstraction to answer the question of this
          % dissertation. Refer to the table of emulation layers in lit review.
          % TEMP

    \item \textbf{Support stochastic noisy input signals.} Inputs into biological
          networks are not clean or sanitised, and there is a level of random noise
          present in them that should be simulated. Random noise in repeated
          experiments makes averages more meaningful and aids in finding reliable results.

    \item \textbf{Simulation should be easily configurable at runtime using
              parameters.} It is important to allow the user of the simulation to
          configure and adapt it to his or her needs using its API surface, without
          reliance on modifying the simulation's internal source code.

    \item \textbf{Networks should be duplicable using standard Python
              techniques.} Once networks have been set up, simulated and
              modified, duplicating a network and its state should be supported
              by the standard Python \texttt{deepcopy} procedure, producing a
              full copy of the network with a fully independent state from the
              original.

    \item \textbf{Experiments performed with the simulation must be repeatable.}
          The results of experiments performed with the simulation must be
          repeatable so they may be verified and their integrity upheld.
          Combined with stochastic inputs to a network, these stochastic inputs
          must be easily seedable to produce deterministic results if required.

          %     \item \textbf{Should be easy to define large groups of neurons with individual parameters drawn from probability distributions}
          %     \item \textbf{Should be capable of probabilistically creating synapses between neuron groups}



          %     \item \textbf{It would be desirable for separate simulations to have the
          %               capability to run in parallel} This requires creating computation pools due
          %           to the limits of the Global Process Lock in the python interpreter.
\end{itemize}