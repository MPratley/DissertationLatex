\chapter{Introduction}
\epigraphhead[50]{
      \epigraph{\large{`There is another world, but it is in this one.'}}{Commonly attrib. Paul Éluard}
}


The topic of creating artificial life has been subject to fervent debate for
hundreds of years, but at the advent of modern computing this conversation took
a turn towards the technological. In his 2005 book “The Singularity Is Near:
When Humans Transcend Biology” author Ray Kurzweil predicts not that we will
create a new species, but that we will be able to transfer our
consciousness to emulations of the human brain.\autocite{kurzweil_singularity_2006}

\begin{quote}
      `The Singularity will allow us to transcend [the] limitations of our
      biological bodies and brains. We will gain power over our fates. Our
      mortality will be in our own hands [\ldots] Although the Singularity has many faces, its most important implication is this: our technology will match and then vastly exceed the refinement and suppleness of what we regard as the best of human traits.'
\begin{flushright}
      \textit{-- Ray Kurzweil \\ The Singularity Is Near: When Humans Transcend Biology
      }
  \end{flushright}
\end{quote}  

The implication of this is
that, assuming in-exhaustive power and processing to maintain the simulation,
humanity would be made immortal. However, there are a number of constraints
technologically that must be reckoned with before this could even be attempted
\parencite{bostrom_whole_2008}, alongside the moral quandaries that are ever
present when discussing the evolution and modification of what it means to be
human.

When discussing brain emulation, we envisage the “uploading” of consciousness,
which in basic terms is the measurement of electrical activity and neural
connections of a brain, and in theory would retain every mental process, keeping
them intact. Using current known techniques, this measurement process would
introduce error in numerous ways; It is perhaps possible that the resolution of
the scanning equipment is not large enough to correctly capture the distances
between neurons, or that changes in the network will occur while measurements
are taken. It is therefore unclear what impact this measurement error would have
on a simulation of the human brain.

\section{Motivation}

This project aims to investigate the effect of measurement error on the accuracy
of a brain simulation with and without synaptic plasticity, and to assess the
feasibility of predicting future brain activity, given the performance of
current imaging techniques.

The primary outcome of this project will be to provide tooling written in Python
that is capable of running simulations in a parallel manner across multiple
threads, that can be modified during runtime to model the effect of measurement
error when `uploading' a brain into a simulation.

\pagebreak

\section{Aims and Objectives}

This project was originally planned to be built on top of the VERTEX simulator.
However, once my research began I decided that I would instead shift more
towards building the simulation tooling myself to better learn and understand
the mathematics than underpin computation neuroscience.

\subsubsection{Analyse and compare three published neural simulation packages.}

Developing an understanding of simulation packages and the models that they
implement will help me inform my design decisions for my own implementation of
this feature. This objective will have been completed when I have summarised and
understood the aims and features of three existing packages.


\subsubsection{Identify the experiments that should be performed to determine
      the relation between measurement error in data from imaging a brain, and the
      performance of a neural simulation of such data.}

In order to prove that that
measurement error has an effect on predicting future brain activity, I will
define experiments that can be performed within the limits of this project. Each
experiment must contribute either to the argument that measurement error has a
major negative impact on the accuracy of a neural simulation, or that
introducing plasticity into the simulation minimises this effect over time.


\subsubsection{Identify the requirements and features that the simulation
      tooling should implement to be capable of performing the project experiments.}

Given the experiments that are defined in order to meet the previous objective,
a set of requirements for the simulation software can be identified. This will
include the Python API presented to the user of the software, required data
structures to store information, and choosing functionality which forms the
minimal viable product. To achieve this, I must have collected enough research
to make conclusive design and functionality decisions.


\subsubsection{Identify
      the parameters in a brain simulation that can be modified during the simulation
      to emulate the measurement error of uploading the human brain.}

Many of the parameters required in the creation of a simulation, such as the
number of layers or the positions and links between neurons could be modified
during a simulation using the tooling this project aims to develop. This
objective will be complete when I have identified these parameters, and
justified why they have been chosen over others.


\subsubsection{Analyse the performance of simulations
      that diverge with different measurement errors from a starting simulation.}

This final objective will assess whether the software designed and developed
during the course of the project can correctly demonstrate the effect of
measurement error on brain simulation. This objective will be met once the
experiments are shown to have provided conclusive results.

\pagebreak

% \begin{itemize} \item \textbf{Analyse and compare three published neural
%     simulation packages.}

%     \item \textbf{Identify the experiments that should be performed to
%               determine the relation between measurement error in data from
%               imaging a brain, and the performance of a neural simulation of
%               such data.}

%     \item \textbf{Identify the requirements and features that the simulation
%               tooling should implement to be capable of performing the project
%               experiments.} \item \textbf{Identify the parameters in a brain
%               simulation that can be modified during the simulation to emulate
%               the measurement error of uploading the human brain.} \item
%               \textbf{Analyse the performance of simulations that diverge with
%               different measurement errors from a starting simulation.}
%               \end{itemize}

% \section{Experiments and Hypotheses}
%

\section{Document Structure}

\subsection*{Background and Literature Review}
This chapter aims to convey the general state of what is considered mainstream
academic and industry opinion, and some of the more cutting edge approaches in
each of the domains that this project touches on.

\subsection*{Planning and Development}
This chapter will discuss the methodology by which project deliverables have
been developed, and explain the rationale of certain design decisions. In
particular, functionality of the neural network simulation will be discussed, as
will the simulation parameters.

\subsection*{Experiments and Results}
The rudimentary correctness of the Python simulation is verified here, and each
of the experiments that form the basis of the arguments made in this
dissertation are expanded on and their results are explained.

\subsection*{Analysis and Evaluation}
An analysis and evaluation will review the results in the previous chapter and
show how the results reflect on the overall project aims. The results of all the
experiments are reviewed holistically and compared with expected findings.
\subsection*{Conclusion}
A summary of the project, and whether the aim and objectives have been met. The
overall success of the project is evaluated and future work that could be
completed in another project expanded on.