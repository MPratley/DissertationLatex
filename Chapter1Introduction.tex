\chapter{Introduction}

The topic of creating artificial life has been subject to fervent debate for
hundreds of years, but at the advent of modern computing this conversation took
a turn towards the technological. In his 2005 book “The Singularity Is Near:
When Humans Transcend Biology” author Ray Kurzweil predicts not that we will
create a new species, but that “The Singularity will allow us to transcend [the]
limitations of our biological bodies and brains”
\parencite{kurzweil_singularity_2006}.
In short, we will be able to transfer our
consciousness to simulations of the human brain. The implication of this is
that, assuming inexhaustive power and processing to maintain the simulation,
humanity would be made immortal. However, there are a number of constraints
technologically that must be reckoned with before this could even be attempted
\parencite{bostrom_whole_2008}, alongside the moral quandaries that are ever
present when discussing the evolution and modification of what it means to be
human. When discussing brain emulation, we envisage the “uploading” of
consciousness, which in basic terms is the measurement of electrical activity
and neural connections of a brain, and in theory would retain every mental
process, keeping them intact. Using current known techniques, this
measurement process would introduce error in numerous ways; It is perhaps
possible that the resolution of the scanning equipment is not large enough to
correctly capture the distances between neurons, or that changes in the network
will occur while measurements are taken. It is therefore unclear what impact
this measurement error would have on a simulation of the human brain.

\section{Motivation}

In this dissertation I hope to answer...

\section{Aims and Objectives}

My project aims to demonstrate the effect of measurement error on brain
simulation by developing tooling and adapting the VERTEX simulator. The outcome
of this project will be to improve the tooling available to run parallel
simulations which diverge at chosen points throughout the runtime of a brain
simulation, with new parameters at the point of divergence. These parameters
will then be modified to model the effect of measurement error when “uploading”
a brain into a simulation.

\begin{itemize}
    \item THESE ARE MY ORIGINAL AIMS AND OBJECTIVES AND NEED TO CHANGE
    \item \textbf{Analyse and compare three published neural simulation packages.}
          Developing an understanding of simulation packages and the models that
          they implement will help me inform my design decisions for my own
          implementation of this feature. This objective will have been
          completed when I have summarised and understood the aims and features
          of three existing packages. Two of these are referenced in my
          Background Research.
    \item \textbf{Analyse and understand the source code and tooling of the
              VERTEX simulator.} It is necessary that before starting
          development, I understand the software design rationale and
          patterns of the VERTEX simulator. In order to achieve this I will
          complete the VERTEX tutorials, on the project website. (Tomsett,
          n.d.). A prerequisite of this is learning the syntax and design
          patterns of MATLAB. This objective will be met when I am working
          on the implementation stage of the project.
    \item \textbf{Identify the requirements of a routine within VERTEX that is
              capable of saving a snapshot of a running simulation.} I will need
          to gather data relating to the functionality and design of a
          simulation snapshot saving system. This will include the API
          presented to the user of the software, required data structures to
          store information, and choosing functionality which forms the
          minimal viable product. To achieve this, I must have collected
          enough research to make conclusive design and functionality
          decisions.
    \item \textbf{Identify the parameters in a brain simulation that can be
              modified during the simulation to emulate the measurement error of
              uploading the human brain.} Many of the parameters required in the
          creation of a simulation, such as the number of layers or the
          positions and links between neurons could be modified during a
          simulation using the tooling this project aims to develop. This
          objective will be complete when I have identified these
          parameters, and justified why they have been chosen over others.
    \item \textbf{Analyse the performance of simulations that diverge with
              different measurement errors from a starting simulation.} This
          final objective will assess whether the software designed and
          developed during the project can correctly demonstrate the effect
          of measurement error on brain simulation. 
\end{itemize}

\section{Document Structure}