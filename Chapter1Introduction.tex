\chapter{Introduction}

The topic of creating artificial life has been subject to fervent debate for
hundreds of years, but at the advent of modern computing this conversation took
a turn towards the technological. In his 2005 book “The Singularity Is Near:
When Humans Transcend Biology” author Ray Kurzweil predicts not that we will
create a new species, but that “The Singularity will allow us to transcend [the]
limitations of our biological bodies and brains”
\parencite{kurzweil_singularity_2006}.
In short, we will be able to transfer our
consciousness to simulations of the human brain. The implication of this is
that, assuming inexhaustive power and processing to maintain the simulation,
humanity would be made immortal. However, there are a number of constraints
technologically that must be reckoned with before this could even be attempted
\parencite{bostrom_whole_2008}, alongside the moral quandaries that are ever
present when discussing the evolution and modification of what it means to be
human. When discussing brain emulation, we envisage the “uploading” of
consciousness, which in basic terms is the measurement of electrical activity
and neural connections of a brain, and in theory would retain every mental
process, keeping them intact. Using current known techniques, this
measurement process would introduce error in numerous ways; It is perhaps
possible that the resolution of the scanning equipment is not large enough to
correctly capture the distances between neurons, or that changes in the network
will occur while measurements are taken. It is therefore unclear what impact
this measurement error would have on a simulation of the human brain.

In this dissertation I hope to answer... 

\begin{itemize}
    \item One gang money
    \item Two gang money
    \item Three gang money
\end{itemize}
