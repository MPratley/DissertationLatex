\chapter{Conclusion}

\epigraphhead[50]{\epigraph{\large{`You can invent someone that was never before
invented'}}{- inspirobot.me\\
A neural network that generates quotes}}

\section{Reflection on personal objectives}

Over the course of researching, developing and writing this dissertation, I have
encountered many ideas, schools of thought, and applications of maths and
computer science that were abstract or unknown to me previously. In choosing a
dissertation that is rooted in computational neuroscience, a topic that is
conceptually new to me, my primary aim was to develop an original understanding
of the limitations that current imaging technology poses to the feasibility of
emulating the whole human brain. 

Throughout the process of creating this dissertation, I have refined my own
researching skills, and am now comfortable synthesising many sources covering a
topic to improve my own understanding, something that I did not have much
practice in beforehand. I have now learned how to effectively cross-reference
hypotheses and conclusions between several, often differing, studies and to
examine as many types of evidence as I can find.



\section{Result reflections and future work}

It seems likely that synaptic plasticity is the key to running simulations of
imperfectly created models that contain error, particularly in synapses where
the precise details of placement and physical composition must be resolved at
the scale of nanometres when imaging. If this is the case, then any future
technology that aims to emulate, and even predict, brain activity must ensure
that its underlying simulation is capable of capturing the nuances of synaptic
plasticity. 

While I believe that the results shown in this dissertation are meaningful, they
do not exclude other interpretations or hypotheses, and there is a lot of work
to be done to better understand the opportunities, potential exploitation, and
limitations of simulation and the requirements of models to be simulated.

Future work should investigate the effect of different Hebbian-Learning models
when applying error to a simulated model, and whether inhibitory synapses have a
similar compensatory effect as excitatory synapses have been shown to in this
dissertation. It could be the case that inhibitory synapses with plasticity have
an effect whereby the equilibrium reached in the network is less stable and
prone to increased deviation from measurement error.

\section{Personal reflection of project experience}
 


% what we know of its function is experimentally determined, not calculated from
% the sequence. It would be wonderful to be able to take a sequence, plug it into
% a computer, and have it spit back a quantitative assessment of all of its
% interactions with other proteins, but we cannot do that, and even if we could, it
% would not answer all the questions we would have about its function, because
% we would
% also need to know the state of all of the proteins in the cell, and the state of
% all of the proteins in adjacent cells, and the state of global and local
% signalling proteins in the environment.
% https://scienceblogs.com/pharyngula/2010/08/17/ray-kurzweil-does-not-understa

During the creation of this dissertation, there have been challenges both in the
research and implementation stages. When researching the mathematical models and
general concepts of computational neuroscience, certain concepts were
intractable for a large period of the early stages of the dissertation, while
other concepts that I considered understood made little sense when the time came
to expand on them for practical application. The time taken to build the
required understanding in the several constituent schools of computational
neuroscience meant that it was difficult to appropriately set the scope of the
project in its infancy. One such change was moving from using an existing
simulation tool in favour of constructing my own. While I believe that creating this tooling was a
necessary exercise in building my understanding of these concepts, modifying an
already existing simulation tool for the experiments in this dissertation would
have enabled faster iteration in creating results as such tooling would already
be optimised for faster simulation.

I am proud of the software and research that has been developed for this
dissertation, and  the progress that I have made in the project time-frame,
particularly with highly disruptive external pressures in the final weeks of the
project. I have found the marriage of the moral, philosophical and technical
concepts that surround brain simulation to be immensely thought provoking.

\section{Final thoughts}

In investigating the effect of measurement error on predicting brain activity in
a computational model, this dissertation takes the synaptic weighting error as
an equivalent approximation to measurement error when imaging the brain. The
simulations developed for, and described in, this dissertation have shown that
this weighting error has a detrimental effect on the performance of a
simulation, where increased deviation from a control simulation was observed in
simulations with emulated measurement error. However, synaptic plasticity
between neurons in a simulation drastically reduces this deviation from the
control simulation.

% In this dissertation, I have investigated the 

% Summarise the lit review

% Summarise the research and project

% Summarise the ethics

% Final sentence

% End with a cool quote

% Singler says that existential despair or existential joy will be the driving force behind the
% design and production of transhumanist technology.
% \autocite{singler_existential_2019}