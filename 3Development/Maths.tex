\section{Modelling mathematics}

In order to program a neural network, I will first enumerate the differential
equations that describe constituent neurons. A neuron state at a given time is
typically defined by a collection of differential equations, while spikes are
occur as a response to changes in this state
\autocite{brette_simulation_2007}.


\subsection{Leaky Integrate and Fire}

The simplest approximation of a spiking neuron that retains the chemical
behaviour of a biological action potential are Leaky Integrate and Fire (LIF) neurons. As any neuron is, in effect, a
signal processing unit, they can be modelled as a circuit.

\begin{figure}
    \begin{subfigure}{.5\textwidth}
        \centering
        \includegraphics[width=.8\linewidth]{figures/images/LIFCircuit1.png}
        \caption{1a}
        \label{fig:sfig1}
    \end{subfigure}%
    \begin{subfigure}{.5\textwidth}
        \centering
        \includegraphics[width=.8\linewidth]{figures/images/LIFCircuit2.png}
        \caption{1b}
        \label{fig:sfig2}
    \end{subfigure}
    \caption{plots of....}
    \label{fig:fig}
\end{figure}

% \begin{figure}[h]
%     \centering
%     \includegraphics{figures/images/LIFCircuit1.png}
%     \DoubleCaption{Schematic diagram of an Integrate and Fire model}
%     {\small{Adapted from \cite{gerstner_spiking_2002}}}
%     \label{LIFCircuit1}
% \end{figure}
% \vspace{1ex}

% The schematic in figure \ref{LIFCircuit1} depicts a 

% The basic circuit of an integrate-and-fire model consists of a capacitor C in parallel with a resistor R driven by a current I(t);

% \begin{figure}[h]
%     \centering
%     \includegraphics{figures/images/LIFCircuit2.png}
%     \caption{Schematic diagram of currents in a Integrate and Fire model}
%     \label{LIFCircuit2}
% \end{figure}
% \vspace{1ex}

\begin{figure*}[h]
    \centering
    \begin{equation}\label{eq:LIF_TC}
        T_C
    \end{equation}
    \begin{equation}\label{eq:LIF_RC}
        \frac{d V}{d t} = -\frac{V}{\tau_L} + \frac{I(t)}{C}
    \end{equation}
    \begin{equation}\label{eq:integ_LIF_RC_VL}
        V(t)= \frac{1}{C} \int_{0}^{t} e^{-\frac{(t-s)}{\tau_L}} I(s) ds
    \end{equation}
    $C$ is the membrane capacity, $g_L$ is the leak conductance and $V_L$ is the leak reversal potential.
    % \caption{Formulae for a Leaky Integrate and Fire Neuron (pre-threshold)}
    \label{LIFequation}
\end{figure*}

\subsection{Spike refractory periods}

\subsection{STDP}