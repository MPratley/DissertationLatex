\documentclass[10.5pt,twoside,openright]{report} 
\usepackage[utf8]{inputenc}
\usepackage[english]{babel}
% \usepackage{verbatim}
% \newcommand\wordcount{\verbatiminput{words.sum}}

% \setlength{\parskip}{\baselineskip}%
\setlength{\parskip}{2mm}%
\setlength{\parindent}{0mm}%

\title{Investigating the effect of measurement error on predicting future brain
activity with a computational model}
\author{Marcus Pratley}
\date{2020}

% % Run texcount on Tex-file and write results to a sum-file
% \immediate\write18{texcount -1 -sum -merge Chapter*.tex -out=words.sum}

\usepackage[ungrad, hyperref]{edmaths}
\usepackage{epigraph}
\usepackage[autostyle]{csquotes}
\usepackage[style=apa]{biblatex} 
\addbibresource{refsv2.bib}

\usepackage[default,light,bold]{sourceserifpro}
\usepackage[T1]{fontenc}

\usepackage{mathtools}
\usepackage{multirow}
\usepackage{amsmath, amsthm, amssymb, amsfonts}
\usepackage{breakcites}
\usepackage{subcaption}
\usepackage{graphicx}
\usepackage{placeins}
\usepackage{sidecap}
\usepackage{mhchem}
% \usepackage{setspace}
% \doublespacing

\usepackage{environ}
\NewEnviron{myequation}{%
\begin{equation}
\scalebox{1.5}{$\BODY$}
\end{equation}
}


% \usepackage{tocloft}
% \usepackage[small]{caption}

\captionsetup[figure]{font=it} 

% some auxiliary lengths for aligning the captions
\newlength\mylena
\newlength\mylenb 

% syntax: \MyCaption{First numbered caption}{Second unnumbered caption}
\newcommand\DoubleCaption[2]{%
  \captionsetup{belowskip=-\baselineskip}
  \settowidth\mylena{\small\itshape\figurename~\thefigure. #1.}
  \settowidth\mylenb{\small\itshape #2.}
  \caption{#1} 
  \caption*{\hspace*{\dimexpr\mylenb-\mylena\relax} (#2)}
  \setlength\belowcaptionskip{\baselineskip}
}

% % settings for the list of figures
% \renewcommand\cftfigpresnum{Figure }
% \renewcommand\cftfigaftersnum{.}
% \newlength\mylend
% \settowidth\mylend{\cftfigpresnum\cftfigaftersnum}
% \addtolength\cftfignumwidth{\mylend}

% \usepackage[bitstream-charter]{mathdesign}
% \usepackage[T1]{fontenc}

\begin{document}

\csname @openrightfalse\endcsname

\maketitle

\declaration

\dedication{With thanks to Frances Hutchins and Professor Marcus Kaiser, who have both been a great help throughout the unusually fraught environment of creating this dissertation. \\ \vspace{1ex} Extra thanks also to my family, and parents in particular, for all their love and support over the course of my degree. }

\begin{abstract}
This paper will discuss the plausibility of accurate computational brain
simulation, given the current limitations of creating digital copies of the
human mind. A spiking neural network has been programmed in the Python programming language, and is used to evaluate the implication of error introduced to a network during simulation. 

My analysis shows the expected correlation between error introduced to a
simulation and the reduction in simulation accuracy, in addition, that the
presence of synaptic plasticity in the network has a mitigating effect on this
reduction in accuracy, increasing the network resilience.

% Modern neural imaging introduces error in numerous ways, TODO
\end{abstract}

\def\table{\def\figurename{Table}\figure}
\let\endtable\endfigure
\renewcommand\listfigurename{List of Figures and Tables}

\tableofcontents
\listoffigures
\csname @openrighttrue\endcsname

\chapter{Introduction}
\epigraphhead[50]{
      \epigraph{\large{`There is another world, but it is in this one.'}}{Commonly attrib. Paul Éluard}
}


The topic of creating artificial life has been subject to fervent debate for
hundreds of years, but at the advent of modern computing this conversation took
a turn towards the technological. In his 2005 book “The Singularity Is Near:
When Humans Transcend Biology” author Ray Kurzweil predicts not that we will
create a new species, but that we will be able to transfer our
consciousness to emulations of the human brain.\autocite{kurzweil_singularity_2006}

\begin{quote}
      `The Singularity will allow us to transcend [the] limitations of our
      biological bodies and brains. We will gain power over our fates. Our
      mortality will be in our own hands [\ldots] Although the Singularity has many faces, its most important implication is this: our technology will match and then vastly exceed the refinement and suppleness of what we regard as the best of human traits.'
\begin{flushright}
      \textit{-- Ray Kurzweil \\ The Singularity Is Near: When Humans Transcend
      Biology (\citeyear{kurzweil_singularity_2006})
      }
  \end{flushright}
\end{quote}  

The implication of this is
that, assuming in-exhaustive power and processing to maintain the simulation,
humanity would be made immortal. However, there are a number of constraints
technologically that must be reckoned with before this could even be attempted
\parencite{bostrom_whole_2008}, alongside the moral quandaries that are ever
present when discussing the evolution and modification of what it means to be
human.

When discussing brain emulation, we envisage the “uploading” of consciousness,
which in basic terms is the measurement of electrical activity and neural
connections of a brain, and in theory would retain every mental process, keeping
them intact. Using current known techniques, this measurement process would
introduce error in numerous ways; It is perhaps possible that the resolution of
the scanning equipment is not large enough to correctly capture the distances
between neurons, or that changes in the network will occur while measurements
are taken. It is therefore unclear what impact this measurement error would have
on a simulation of the human brain.

\section{Motivation}

This project aims to investigate the effect of measurement error on the accuracy
of a brain simulation with and without synaptic plasticity, and to assess the
feasibility of predicting future brain activity, given the performance of
current imaging techniques.

The primary outcome of this project will be to provide tooling written in Python
that is capable of running simulations in a parallel manner across multiple
threads, that can be modified during runtime to model the effect of measurement
error when `uploading' a brain into a simulation.

\pagebreak

\section{Aims and Objectives}

This project was originally planned to be built on top of the VERTEX simulator.
However, once my research began I decided that I would instead shift more
towards building the simulation tooling myself to better learn and understand
the mathematics than underpin computation neuroscience.

\subsubsection{Analyse and compare three published neural simulation packages.}

Developing an understanding of simulation packages and the models that they
implement will help me inform my design decisions for my own implementation of
this feature. This objective will have been completed when I have summarised and
understood the aims and features of three existing packages.


\subsubsection{Identify the experiments that should be performed to determine
      the relation between measurement error in data from imaging a brain, and the
      performance of a neural simulation of such data.}

In order to prove that that
measurement error has an effect on predicting future brain activity, I will
define experiments that can be performed within the limits of this project. Each
experiment must contribute either to the argument that measurement error has a
major negative impact on the accuracy of a neural simulation, or that
introducing plasticity into the simulation minimises this effect over time.


\subsubsection{Identify the requirements and features that the simulation
      tooling should implement to be capable of performing the project experiments.}

Given the experiments that are defined in order to meet the previous objective,
a set of requirements for the simulation software can be identified. This will
include the Python API presented to the user of the software, required data
structures to store information, and choosing functionality which forms the
minimal viable product. To achieve this, I must have collected enough research
to make conclusive design and functionality decisions.


\subsubsection{Identify
      the parameters in a brain simulation that can be modified during the simulation
      to emulate the measurement error of uploading the human brain.}

Many of the parameters required in the creation of a simulation, such as the
number of layers or the positions and links between neurons could be modified
during a simulation using the tooling this project aims to develop. This
objective will be complete when I have identified these parameters, and
justified why they have been chosen over others.


\subsubsection{Analyse the performance of simulations
      that diverge with different measurement errors from a starting simulation.}

This final objective will assess whether the software designed and developed
during the course of the project can correctly demonstrate the effect of
measurement error on brain simulation. This objective will be met once the
experiments are shown to have provided conclusive results.

\pagebreak

% \begin{itemize} \item \textbf{Analyse and compare three published neural
%     simulation packages.}

%     \item \textbf{Identify the experiments that should be performed to
%               determine the relation between measurement error in data from
%               imaging a brain, and the performance of a neural simulation of
%               such data.}

%     \item \textbf{Identify the requirements and features that the simulation
%               tooling should implement to be capable of performing the project
%               experiments.} \item \textbf{Identify the parameters in a brain
%               simulation that can be modified during the simulation to emulate
%               the measurement error of uploading the human brain.} \item
%               \textbf{Analyse the performance of simulations that diverge with
%               different measurement errors from a starting simulation.}
%               \end{itemize}

% \section{Experiments and Hypotheses}
%

\section{Document Structure}

\subsection*{Background and Literature Review}
This chapter aims to convey the general state of what is considered mainstream
academic and industry opinion, and some of the more cutting edge approaches in
each of the domains that this project touches on.

\subsection*{Planning and Development}
This chapter will discuss the methodology by which project deliverables have
been developed, and explain the rationale of certain design decisions. In
particular, functionality of the neural network simulation will be discussed, as
will the simulation parameters.

\subsection*{Experiments and Results}
The rudimentary correctness of the Python simulation is verified here, and each
of the experiments that form the basis of the arguments made in this
dissertation are expanded on and their results are explained.

\subsection*{Analysis and Evaluation}
An analysis and evaluation will review the results in the previous chapter and
show how the results reflect on the overall project aims. The results of all the
experiments are reviewed holistically and compared with expected findings.
\subsection*{Conclusion}
A summary of the project, and whether the aim and objectives have been met. The
overall success of the project is evaluated and future work that could be
completed in another project expanded on.
\chapter{Background and Literature Review}

\epigraphhead[50]{\epigraph{\large{`The brain computes!'}}{Biophysics of Computation\\
by Christof Koch}}

\section{Scope of this review}
This literature review provides a high level overview of the core topics relating to this dissertation, and provides context for the requirements and assumptions of the software development involved in this project. 

If one were to consider a brain as analogous to a computer, with the memories of
the brain as the hard drive, it is these memories that an  emulation of a brain
would require to be considered a continuation of a being
\autocite{eichenbaum_cognitive_2011}. The academic schools that study and debate
the inner workings and logistics of the human mind are many in number, and it is
outside the scope of this project to summarise all of them in detail.  This
literature review is therefore concerned with the replication of the neural
structure and its constituent function at a low level.

% TODO: Levels of emulation pg.13 Bostrom and Sandburg Roadmap

\section{Spiking neural networks}

\begin{quote}
    `The brain computes! This is accepted as a truism by the majority
    of neuroscientists engaged in discovering the principles employed in the
    design and operation of nervous systems. What is meant here is that any
    brain takes the incoming sensory data, encodes them into various biophysical
    variables, such as the membrane potential or neuronal firing rates, and
    subsequently performs a very large number of illspecified operations,
    frequently termed computations, on these variables to extract relevant
    features from the input.'
    \begin{flushright}
        \textit{-\autocite{koch_biophysics_2004}}
    \end{flushright}
\end{quote}

The mammalian nervous system is principally composed of neurons and glial cells.
Glial cells primarily perform supplementary roles in the function and processing
of data in the nervous system, but it is still not fully understood what further
role they play \autocite{walz_role_1989}. Neurons are less numerous in the
nervous system, and are chemically unique in their functioning in the body.

\begin{figure}[t]
    \centering
    \includegraphics{figures/graphs/huxhog_spike.png}
    \caption[The typical form of a neuronal action potential]{The typical form
    of a neuronal action potential, where the potential difference between the
    resting and peak potentials is approx. $40-100 mV$.
        \\\small{Created by
    Nir.nossenson@wikipedia.com, licenced 
            \href{https://creativecommons.org/licenses/by-sa/4.0/deed.en}{CC BY-SA 4.0}}}
    \label{neuronalactionpotentialexample}
\end{figure}
\vspace{1ex}

Each neuron acts as a gate of sorts, holding or releasing its potential in
response to signals as part of a greater network of neurons. This release often
takes the form of a spike, as seen in figure \ref{neuronalactionpotentialexample}, depicting a typical neuronal action potential of a neuron in the nervous system.
Together, the neurons in these networks perform the signal processing and
routing that "computes" the many sensory inputs to the nervous system
\autocite{koch_biophysics_2004}. While not all neurons are spiking neurons, it %TODO percentage of neurons that are spiking?
is the properties of spiking neurons and the interactions between them that are
the focus of this review and ultimately this dissertation.

\subsection{Hodgkin–Huxley model}

In 1952, Alan Hodgkin and Andrew Huxley described a model of the squid giant
axon that accurately represents both the chemical changes and the form of the
action potential spike in the axon. \autocite{hodgkin_quantitative_1952}

The strength of the Hodgkin–Huxley model is in its adaptability. Every change in
a biological neuron during a spike, be it chemical or physical, can be mapped
across to changes in the Hodgkin–Huxley model, and more complex behaviour
can be recreated in the model thanks to its basis in the core chemistry of a
neuron. However, this complexity is not without pitfalls as the computational
power required to simulate such advanced models is so large,  that it is no
longer feasible to simulate it within a network.

\subsection{Faster neuron models for large scale computation}

While it can be desirable to accurately model and simulate the chemical
interactions between individual components of the nervous system, it is
computationally infeasible to do so on a scale similar to that of a real-world
system due to the computational complexity of the Hodgkin-Huxley equations.
Fortunately, as a network becomes larger, the exact mechanisms of its individual
components can be reliably approximated, and faster activation or threshold functions
that perform similarly to real ones can be substituted.

One such example of a simple model of a spiking neuron is the Leaky Integrate
and Fire (LIF) neuron. It uses a simple threshold function to approximate the
action potential spike generation of a Hodgkin–Huxley neuron
\autocite{trappenberg_fundamentals_2009}. As this is a relatively simple
approximation, it is tempting to assume that such a model would prove inaccurate
for anything more than basic prototypes, however LIF models have been shown to
replicate spiking patterns of more complex models with a low margin of error,
provided the models are under natural conditions
\autocite{teeter_generalized_2018}. This dissertation will go into more detail on
the equations and implementation of LIF neurons in section
\ref{Modellingmathematics} of this document.

\subsection{Hebbian learning and synaptic plasticity}

In biological networks, neurons adapt their relationships over time in reaction
to new environments or long term changes, in a process called "Hebbian learning"
\autocite{trappenberg_fundamentals_2009}. This process can be replicated in a
spiking neural network through adding a vector of weights to each neuron that
determines the relative connection strength of each pre synaptic input. These
weights will change over time according to the observed correlation between
pre-synaptic and post-synaptic spikes, forming Spike-Time Dependent Plasticity
(STDP) \autocite{iakymchuk_simplified_2015}. The concept and implementation of STDP is expanded on in chapter 3, starting with
equation \ref{eq:stdpdw}.



\section{Computational models for brain simulation}

The coupled differential equations that define a compartmental model of a neuron
can be solved numerically, and can therefore be programmed in a high-level
programming language relatively simply. However, while programmatically solving
differential equations is straightforward in principle, the computational power
and provision of memory required makes simulating large networks of neurons
difficult to do in practice \autocite{trappenberg_fundamentals_2009}. Existing
simulation packages provide a foundation from which biological experiments can
be conducted, and aim to abstract away purely computational concerns. Some of
these packages that have different approaches to solving this are described in
more detail below.

\subsection{TODO: Rate vs Event Models}
write a section on rate vs event models. Brian docs have a good page on this.
% https://brian2.readthedocs.io/en/stable/advanced/how_brian_works.html#clock-driven-versus-event-driven

\subsection{Available neural simulation and modelling software packages}

\subsubsection{VERTEX}
The Virtual Electrode Recording Tool for EXtracellular potentials (VERTEX) is a
simulator for large networks of neurons written in the MATLAB programming
language. More specifically, it aims to reproduce the measurements that are
produced by in-vivo recordings from patients fitted with multi-electrode
arrays. This is achieved by locating each neuron in 3D space, placing virtual
electrodes in the same space, and calculating the change in potential at each
stage of the simulation at each electrode. The neuron model used in VERTEX is
also more advanced than similar models described in this section in that the
local field potential (LFP) of the neuron is spatially realistic
\autocite{tomsett_virtual_2015}. A major advantage of this design is that it
becomes easy to map between general neuronal activity in a simulated model and
brain activity of, say, a real-world patient in a medical trial.

A 2019 update to VERTEX introduced a range of simulation features, the most
notable of which for the purposes of this project being Spike Time Dependant
Plasticity (STDP), which dynamically adapts the weighting of synapses based on
the observed causality of pre-synaptic spikes on post-synaptic spikes
\autocite{thornton_virtual_2019}. Details on how this may be implemented and
used are covered further in Chapter 3.

\begin{figure}[h]
    \centering
    \includegraphics{figures/graphs/coresVERTEX.png}
    \DoubleCaption{Impact of parralel comIn chapter 3 I will putation on
        simulation performance in VERTEX} {\cite{tomsett_virtual_2015}}
    \label{VERTEXparallel}
\end{figure}
\vspace{1ex}

VERTEX itself is multithreaded, and makes use of the compute pooling feature in
MATLAB to separate groups of tasks into separate logical threads. This means
that, with a sufficiently large model, doubling the number of available workers
in the pool nearly halves the time taken to simulate, as shown in figure
\ref{VERTEXparallel} above.

\subsubsection{NEURON}

NEURON describes itself as ``a tool designed specifically for solving the
equations that describe nerve cells'' and provides a holistic environment for
developing and simulating models of individual neurons and neural networks. It
is particularly adept at modelling complex chemistry in neuron models, and can
be used when the field potential of a location very close to a neuron needs to
be accurately modelled \autocite{carnevale_neuron_2006}. This does, however,
mean that simulating large networks is computationally expensive, and more
complex models require correctly tuned parameters for each new feature which
increases the time setting up simulations.

Further simulation toolkits may be built atop NEURON, and one such example is
LFPy. LFPy is a Python tool using NEURON that aims to accurately provide
extracellular potential readings for single neuron models
\autocite{hagen_hybrid_2016}. It can also provide extracellular potential
readings for networks in the most recent versions \autocite{hagen_lfpy_2019},
however the caveat that NEURON based networks are limited in size by available
resources still holds.

\subsubsection{Brian}

Brain is another simulator for spiking neural networks, distributed as a python
library. Brian offers a unique programming interface whereby the differential
equations that define a Neuron can be written in plain-text and are interpreted
at runtime. Provided the equation can be interpreted, this approach
makes computing mathematical models extremely simple, and is ideal for
quickly testing variations of a model \autocite{stimberg_brian_2019}. This is,
however, less useful for more complex relationships between neurons where the
plain-text mathematical syntax develops its own API on top of the Python API.
Fortunately users of the library may provide native python functions to define
the behaviour of their neurons if desired \autocite{noauthor_functions_2020}.
The complexity of actual neurons in a Brian simulation is as complex as a
programmer intends it to be: Neurons can be placed in physical space if desired,
but this is optional. The time taken to simulate a network is largely dependant
on the time complexity of the user-defined equations that define the neurons.

\subsection{Similarities in existing software packages}

Each of the software packages described above is capable of doing simulating a
network of Neurons, but each has a distinct set of goals. Some, like Brian, are
intended as tools to study the spiking patterns and features of generic neural
networks, while VERTEX provides a more specialised tool that is optimised for
measuring the observed potential changes in a similar manner to capturing
experimental data. These differences are also seen in the API available to
developers
and the distribution methods that each package uses. VERTEX requires that the
user defines the parameters for Neuron grouping and placement while abstracting
away the exact definition of the Neuron behaviour, while NEURON and Brian are
very much focussed on the Neuron definition, and space placement and recording
mechanisms are functions largely left to the programmer to define.
\section{Imaging and Image Processing}

Considering prediction of brain activity requires that we look not only at the
current state of brain simulation, but also towards the future. A Biological
brain processing the life of an organism, and a simulation of the same requires
a link to made between the two: the state of the brain must be copied over as a
snapshot. This imaging process would likely take the form of an in-vitro scan of
a preserved nervous system or some form of in vivo scan. These images would be
processed and turned into a model that could be simulated. Current techniques for imaging the brain give us some insight to how this
uploading process might function and perform.

\subsection{Current Methods of Scanning the brain}

Given a 2D substrate such as a layer of brain
tissue, there are several methods of producing a high resolution image. These
methods typically balance the required resolution and image noise with the speed
and scale with which the image is created.

\begin{figure}[h]
    \centering
    \includegraphics{figures/graphs/scaleexample.png}
    \DoubleCaption{Image depicting the scale of the the brain.}
    {TEMPORARY IMAGE}
    \label{scaleexample}
\end{figure}
\vspace{1ex}

Given the level of emulation that is intended from a model, the images that
synthesise the model must resolve a minimum level of detail. In figure
\ref{scaleexample}, the general position of axons(??) and rough links between
them can be identified, but more subtle morphologies of these links are missing.
Another imaging method could resolve further detail and allow for such
morphologies to be processed by a simulation. The specifications and drawbacks
of such imaging methods are described below.

\subsubsection*{MRI Scanning}

- Common procedure
- Enough detail for general structure, but provides little to no insight to fine
details such as placement of dendritic!Typo?! spines or the exact size or placement of
synapses between neurons.
- As the patient is typically alive during a scan, there will be significant
measurement drift as the current is still moving around the brain as normal.

\subsubsection*{XRay}

- Not naturally useful, have to inject a substrate into the brain for this to
create an image of any meaningful contrast ?!?!CITE
- Variants exist which provide much higher resolution but at the cost of time 

\subsubsection*{Electron Microscope Scanning}

- Very precise
- Can take a very long time
- Need well "frozen" slices of the brain for this to be of any use or you'll
have major measurement drift as the brain decays.

\subsection{Turning image data into a connectome}

\begin{quote}
    Dense connectomic mapping of neuronal circuits is limited by the time and
    effort required to analyze 3D electron microscopy (EM) datasets. Algorithms
    designed to automate image segmentation suffer from substantial error rates
    and require significant manual error correction. Any improvement in
    segmentation error rates would therefore directly reduce the time required
    to analyze 3D EM data.
    \autocite{pallotto_extracellular_2015} 

    The delineation of morphologies has proven to be the most difficult step to
    automate. Current machine learning-based analysis methods enable
    semi-automated reconstructions (segmentations) of neurons that still require
    significant human effort to correct.
    \autocite{helmstaedter_connectomic_2013}
\end{quote}

\subsection[Error induced through noise]{Examination of Error induced through the imaging process}

Prediction of future brain activity through simulation requires an accurate and
detailed connectome of a brain, with synapses and neurons correctly located in
space.\autocite{bostrom_whole_2008} The accuracy of such a model depends on the
resolution of the imaging method used to create it. The error resulting from
such a imaging method is the measurement error. Depending on imaging procedure, brain matter may shift in composition during the course of the scan, which is the cause of measurement drift, itself a form of measurement error.

\setlength{\tabcolsep}{3.3ex}
\renewcommand{\arraystretch}{1.1}
\begin{table}[h!]
    \centering
    \addtolength{\leftskip} {-0.5cm}
    \addtolength{\rightskip}{-0.5cm}
    \begin{tabular}{@{}lllll@{}}
        Method              & Environment & Resolution ($\mu m$ )                & Time       &
        Error (approx. \%)                                                                \\
        \hline
        MRI                 & In Vivo     & 6                  & 30 minutes &
        95                                                                          \\
        MRI microscopy      & In Vitro    & 3                   & -          &
        85                                                                          \\
        XRay microscopy     & In Vitro    & 0.03                       & -          &
        30                                                                          \\
        Electron microscopy & In Vitro    & \textasciitilde 0.03-0.001 & > 3 months &
        <1                                                                          \\
        Theoretical Ideal   & Either      & <0.005                       & <500s
        & <1
        \\
        \hline
    \end{tabular}
    \caption[Table comparing brain imaging methods.]{Table comparing brain
    imaging methods. Time taken and error is extrapolated over the whole human
    brain. Error is approximated from size of dendritic spines. Data primarily
    sourced from \autocite{bostrom_whole_2008, kaynig_large-scale_2015}}
    \label{imagemethodcomparison1}
\end{table}
\setlength{\tabcolsep}{1ex}


\section{Ethics in Brain Simulation}

In considering with the concept of whole brain emulation, one must ensure that
standard medical and experimental ethical requirements are being followed.
However, if the cognitive capability of the emulated brain is proven to be equal
to that of the biological, one must also define the ethical and moral boundaries
that stem from the creation of possibly sentient artificial life. 

\subsection{Ethics in neuroscience}

In a 2012 essay on the ethics of non-invasive brain stimulation (NIBS), Roi
Cohen Kadosh et al. argue that there is a temptation in the broader neuroscience
community to use experimental techniques to treat neurological conditions that
are poorly understood, especially in children, where these techniques have shown
promise but have ill-defined or unknown side effects. They advocate for careful
and considered evaluation of the effects of NIBS on the general population but
children in particular, proceeding cautiously with trials afterwards. One key
takeaway is that it is vital for participants of such trials to be found from
the widest possible socio-economic and racial backgrounds
\autocite{kadosh_neuroethics_2012}. 

Meanwhile, attempts to build an ethical consensus around human cognitive
advancements have long been subject to heated discourse, particularly in the
field of performance enhancing drugs. In an article published in 2008, Henry
Greely et al. argue the following:

\begin{quote}
    [Cognitive performance altering drugs], along with newer technologies such
    as brain stimulation and prosthetic brain chips, should be viewed in the
    same general category as education, good health habits, and information
    technology — ways that our uniquely innovative species tries to improve
    itself.
    \begin{flushright}
        -\textit{\autocite{greely_towards_2008}}
    \end{flushright}
\end{quote}

In essence, that it is a moral duty of public health and the scientific
community to explore all and any avenues available to us when it comes to
advancing the intelligence or performance of humans in society. However, it is
also acknowledged that cognitive alterations, such as neural stimulation or
mind-altering drugs are an inherently invasive procedure, and that consent must
be informed and well judged. 

% TODO morals


% \section{Deductions from this literature review}

% \textsc{TODO Write something here}

\chapter{Design, Development and Methodology}

In order to better understand the fundamentals of computational neuroscience, I
am taking a "from scratch" approach to building this project. While this
increases the time required to run simulation experiments, this approach allows
me to build confidence in my understanding of the maths and models that are
commonly used in the wider computational neuroscience community.

\section{Simulation Requirements}

    % Justification should cite imaging requirements and state that this is an
    % appropriate level of abstraction to answer the question of this
    % dissertation. Refer to the table of emulation layers in lit review.

While there are many different levels of emulation possible in the field of
computational neuroscience, for the purposes of this dissertation, a spiking
neural network is appropriate. This allows inserting error into the system that
is equivalent to imaging error in both neurons and synapses when constructing
connectomes from EM images. It is, however, not necessary to precisely emulate
the chemical processes within the nervous system and these processes may be
mathematically approximated. The following are the software requirements of the
simulation that I have developed for the research purposes of this dissertation.

\begin{itemize}
    \item \textbf{Use a clock-driven model for Simulation} where time interval
          is small enough to minimise descritization errors.

    \item \textbf{Support some form of plasticity between neurons.} This should
          be capable of performing weight updates in an unsupervised manner manner
          across the network.
          % [TEMP] Justification should cite imaging requirements and state that this is an
          % appropriate level of abstraction to answer the question of this
          % dissertation. Refer to the table of emulation layers in lit review. TEMP
    \item \textbf{Simulation should be easily configurable at runtime using parameters}
    \item \textbf{Should be easy to define large groups of neurons with individual parameters drawn from probability distributions}
    \item \textbf{Should be capable of probabilistically creating synapses between neuron groups}
    \item \textbf{Experiments performed with the simulation must be repeatable}
          hence the stochastic element of network generation should be seed-able so
          networks can be recreated.
    \item \textbf{Network should support an optional stochastic input to emulate noise}
    \item \textbf{Networks should be duplicable using standard Python techniques}
    \item \textbf{It would be desirable for separate simulations to have the
              capability to run in parallel} This requires creating computation pools due
          to the limits of the Global Process Lock in the python interpreter.
\end{itemize}
\section{Modelling mathematics}

TODO: Fix this paragraph
In order to program a neural network, I will first enumerate the differential
equations that describe constituent neurons. A neuron state at a given time is
typically defined by a collection of differential equations, while spikes are
occur as a response to changes in this state
\autocite{brette_simulation_2007}.


\subsection{Leaky Integrate and Fire}

The simplest approximation of a spiking neuron that retains the chemical
behaviour of a biological action potential are Leaky Integrate and Fire (LIF) neurons. As any neuron is, in effect, a
signal processing unit, they can be modelled as a circuit.

\begin{figure}[h]
    \begin{subfigure}{.5\textwidth}
        \centering
        \includegraphics[width=.9\linewidth]{figures/images/LIFCircuit1.png}
        \caption{IF model}
        \label{fig:LIFSchemA}
    \end{subfigure}%
    \begin{subfigure}{.5\textwidth}
        \centering
        \includegraphics[width=.7\linewidth]{figures/images/LIFCircuit2.png}
        \caption{IF Currents}
        \label{fig:LIFSchemB}
    \end{subfigure}
    \DoubleCaption{Schematics of an Integrate and Fire model}{\small{Redrawn and
    adapted from \cite{gerstner_spiking_2002}}}
    \label{fig:LIFSchem}
\end{figure}

The schematic in figure \ref{fig:LIFSchemA} depicts a simple integrate-and-fire
circuit; A capacitor $C$ holds $q$ charge, and is in parallel with a resistor
$R$, both driven by the input current $I(t)$. $V_{max}$ represents an arbitrary
threshold function, which upon firing will release the potential of the
capacitor as a `spike'. 

We can arrange the formulae for current and capacitance to find the membrane
potential $v(t)$ at time $t$. The input current $I(t)$ is split between the two
components of the circuit, and as such can be calculated by summing the current
over them, $I(t) = I_R + I_C$. This is shown in figure \ref{fig:LIFSchemB}.
Through Ohm's law, $I_R = \frac{v(t)}{R}$, while from the definition of capacity,
the current on the capacitor is found as 
$I_C = C \frac{d v(t)}{d t}$ \autocite{gerstner_spiking_2002}. 
 
\begin{equation}\label{eq:IF_Itpre}
    I(t) = I_R + I_C = \frac{v(t)}{R} + C \frac{d v(t)}{d t}
\end{equation}

Multiplying $I(t)$
by $R$ gives equation \ref{eq:IF_It}.

\begin{equation}\label{eq:IF_It}
    R I(t) = v(t) + R C \frac{d v(t)}{d t}
\end{equation}

Defining $RC$ as time constant $\tau_C$, rearranging \ref{eq:IF_It} will
give the standard form for an IF neuron, equation \ref{eq:IF_SF}.

\begin{equation}\label{eq:IF_SF}
    \tau_C \frac{d v(t)}{d t} = - v(t) + R I(t)  
\end{equation}

This can be easily written in Python to test if find the spiking pattern we
expect, given an arbitrary potential threshold.




% \begin{figure*}[h]
%     \centering

% \end{figure*}



% \begin{figure*}[h]
%     \centering
%     \begin{equation}\label{eq:LIF_TC}
%         T_C
%     \end{equation}
%     \begin{equation}\label{eq:LIF_RC}
%         \frac{d V}{d t} = -\frac{V}{\tau_L} + \frac{I(t)}{C}
%     \end{equation}
%     \begin{equation}\label{eq:integ_LIF_RC_VL}
%         V(t)= \frac{1}{C} \int_{0}^{t} e^{-\frac{(t-s)}{\tau_L}} I(s) ds
%     \end{equation}
%     $C$ is the membrane capacity, $g_L$ is the leak conductance and $V_L$ is the leak reversal potential.
%     % \caption{Formulae for a Leaky Integrate and Fire Neuron (pre-threshold)}
%     \label{LIFequation}
% \end{figure*}

\subsection{Spike refractory periods}

\subsection{STDP}

\section{Transferring models to Python}

\begin{figure}[h]
    \centering
    \addtolength{\leftskip} {-4cm}
    \addtolength{\rightskip}{-4cm}
    \includegraphics[width=1.3\linewidth]{figures/images/workflow.png}
    \caption{The simulation workflow}
    \label{fig:workflow}
\end{figure}

In order to flexibly mix object oriented and functional programming paradigms,
I've chosen to develop these models in the Python programming language. Python
is home to a healthy and active ecosystem of libraries and development practices
for scientific and statistical research to draw from. Of particular note are the
SciPy libraries, which provide a performant set of functions and objects that
speed up floating point arithmetic and vector mathematics. 

In figure \ref{fig:workflow}, the planned workflow for using the simulation is
mapped out, where the user can easily mix and match large probabilistically
generated networks and manually configured groups of neurons. It is in the
simulate and modify loop that experimentation can take place.

\subsection{Object oriented neurons}

In order to make the simulation as extensible as possible, I have chosen to
implement it in an object oriented manner, where functionality of the components
in a the neural network is compartmentalised to the objects that should own said
functionality. For instance, each neuron holds its state, and state history, and
the functions that modify that state.

Another significant advantage of using object oriented programming in Python is
that objects can be deeply cloned using Python's \texttt{copy} library, a significant benefit when it comes to
duplicating networks. 

\begin{figure}[h!]
    \centering
    % \addtolength{\leftskip} {-4cm}
    % \addtolength{\rightskip}{-4cm}
    \includegraphics[width=0.5\linewidth]{figures/images/UML2.png}
    \caption{UML Diagram}
    \label{fig:uml}
\end{figure}

On a basic level, the hierarchy within the simulation is as follows: All neurons
and synapses belong to a Simulation object, and the neurons in the network are
doubly-linked with synapses in-between. This is illustrated in figure
\ref{fig:uml}. Neurons have been split into a base class \texttt{AbstractNeuron}
that holds the vector of input synapses and presents an interface to get the
neuron potential, but the specific implementation details are left to
\texttt{AbstractNeuron}'s implementations. One of these implementations is the
LIF neuron described earlier in this chapter, and the other is
\texttt{NoisySine}, a Neuron that uses overlapping sine-waves to generate an
output current. 

\subsection{Network Topology}

In order to model the passing of information through the simulation, there were
several approaches that I could take. In order to pick a solution, it is
necessary to map out the relationships between neurons in some sample topologies
and the features that need to be supported by the synapses in such topologies.

\subsubsection{Stochastic generation of networks}

When creating large networks of neurons, it is simpler to organise them in
groups that are generated with random variables, and to stochastically link
neuron groups with synapse groups. 

\subsubsection{Function parameters instead of static parameters}

It is desirable to make the parameters that define the random distribution in a group
generation function be configurable at runtime, but increasing the number of
parameters in a function signature increases the mental overhead when a programmer is
using an API [CITE THIS]. 

However,
this can be simplified even further by passing a single PYTHON CALLABLE, which
in practice is a closure that encapsulates a function that produces a
distribution of the user's choice. 
% TODO
% \subsection{Measuring spiking patterns across the network}

% -Mark certain neurons as recording
% -This can be done as a boolean flag on the neuron to tell it to keep a time
% trace of its potential, or by inserting measurement objects into the simulation.
% These measurement objects can take the form of 

\section{Comparing spiking patterns between two networks.}
\label{Comparingspikingpatternsbetweentwonetworks}

\subsection{Hamming distance}
Due to the descritization of time required in clock based simulation, spikes at
each neuron are aligned to a "grid" of sorts. Therefore, given two binary
vectors holding the spiking state of two neurons, $S_{arr}$ and $S^\prime_{arr}$, the bit at a given index $i$ represents
the spike state at the same discrete point in simulation time in either vector.
With this in mind, it is therefore possible to calculate the number of bits
that must be flipped in $V_{arr}$ for it to contain the same sequence of bits as
$V^\prime_{arr}$. The resulting number is the Hamming Distance. While the
Hamming distance can be used to observe the change in spiking pattern between
two neurons, it is limited as it does not provide a spatial metric of change
between two sequences. Another limitation is derived from its simplicity, as
spiking patterns tell us little about the membrane potential of a neuron leading
up to a spike.

\subsection{Kullback-Leibler divergence}

Where the Hamming Distance provides a method by which binary spiking trains of a
neuron can be compared, Kullback-Leibler (KL) divergence provides a similar
method of comparison for the distribution of potential over time between two
neurons. Comparing potential does have a caveat however: if the total integrated
sum of potential is different between the two neurons, the KL divergence is not
a true measure of comparison. This can be mitigated by running the same
experiment multiple times, while experiment variables are changed in a
probabilistic manner. The average divergence could then be calculated. KL
divergence shares a major caveat with Hamming Distance in that it does not
measure distance. This is best illustrated by figure
\ref{fig:distributioncompKLEMD}, where the large change in distance between
distributions within the sub-figures has no effect on the KL divergence. Also of
note is that divergence is not a symmetrical metric, so $KL(A, B) \neq KL(B,
A)$, so any comparisons of KL divergence must ensure that the same divergence is
being compared.

\begin{figure}[h!]
    \begin{subfigure}{.5\textwidth}
        \centering
        \includegraphics[width=1.0\linewidth]{figures/graphs/EMDvsKLclose.eps}
        \caption{Close distributions}
        \label{fig:closedist}
    \end{subfigure}%
    \begin{subfigure}{.5\textwidth}
        \centering
        \includegraphics[width=1.0\linewidth]{figures/graphs/EMDvsKLfar.eps}
        \caption{Distant distributions}
        \label{fig:fardist}
    \end{subfigure}
    \caption[Comparison between Earth Mover Distance and Kullback-Leibler Divergence]{Comparison between Earth Mover's Distance and Kullback-Leibler Divergence on two pairs of distributions with the same integral (area). The closeness of the distributions is reflected in a lower EMD while KL divergence remains the same. EMD is therefore a more intuitive metric to compare distributions.}
    \label{fig:distributioncompKLEMD}
\end{figure}

\FloatBarrier

\subsection{Earth Mover's Distance (EMD)}

Also known as the `Wasserstein Distance', Earth Mover's Distance is, simply put,
given two different mounds of earth, the amount of earth that would need to be
moved from one to the other such that both mounds are the same. More formally,
given two distributions in N-dimensional space $P, Q$, the Earth Mover's
Distance is the N-dimensional distance between them \autocite{pele_fast_2009}. While more computationally
expensive to calculate than Kullback-Leibler divergence, EMD is symmetric, and
more importantly, is aware of relative space between data-points when comparing
distributions. This comparison is shown in figure
\ref{fig:distributioncompKLEMD}, where a change in distribution distance between
the two sub-figures changes the total EMD between them.
% % \chapter{Experiments and Results}

% \section{The effect of Gaussian error on weights between spiking neurons}
% - Explanation of Experiment
% - Graph
% - Summary of Graph
% \section{The effect of STDP on weighting errors between spiking neurons}
% - Explanation of Experiment
% - Graph
% - Summary of Graph
% \section{The effect of network size on weighting errors between spiking neurons}
% - Explanation of Experiment
% - Graph
% - Summary of Graph

\chapter{Analysis and Evaluation}

\section{What do the results mean?}

\section{Are they successful? Are they expected?}

\section{Personal reflection of result success/lack thereof}

\section{Personal reflection of whole project experience}
 - Certain concepts were intractable for quite some time 
 - concepts that I considered understood made little sense when the time came to
 develop on them
\chapter{Conclusion}

\section{What went well?}

\section{What could have been improved?}

\section{Have I met all my objectives?}

End with a cool quote

\printbibliography[heading=bibintoc]

\end{document}
