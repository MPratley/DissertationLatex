\section{Ethics in Brain Simulation}

In considering with the concept of whole brain emulation, one must ensure that
standard medical and experimental ethical requirements are being followed.
However, if the cognitive capability of the emulated brain is proven to be equal
to that of the biological, one must also define the ethical and moral boundaries
that stem from the creation of possibly sentient artificial life. 

\subsection{Ethics in neuroscience}

In a 2012 essay on the ethics of non-invasive brain stimulation (NIBS), Roi
Cohen Kadosh et al. argue that there is a temptation in the broader neuroscience
community to use experimental techniques to treat neurological conditions that
are poorly understood, especially in children, where these techniques have shown
promise but have ill-defined or unknown side effects. They advocate for careful
and considered evaluation of the effects of NIBS on the general population but
children in particular, proceeding cautiously with trials afterwards. One key
takeaway is that it is vital for participants of such trials to be found from
the widest possible socio-economic and racial backgrounds
\autocite{kadosh_neuroethics_2012}. 

Meanwhile, attempts to build an ethical consensus around human cognitive
advancements have long been subject to heated discourse, particularly in the
field of performance enhancing drugs. In an article published in 2008, Henry
Greely et al. argue the following:

\begin{quote}
    [Cognitive performance altering drugs], along with newer technologies such
    as brain stimulation and prosthetic brain chips, should be viewed in the
    same general category as education, good health habits, and information
    technology — ways that our uniquely innovative species tries to improve
    itself.
    \begin{flushright}
        -\textit{\autocite{greely_towards_2008}}
    \end{flushright}
\end{quote}

In essence, that it is a moral duty of public health and the scientific
community to explore all and any avenues available to us when it comes to
advancing the intelligence or performance of humans in society. However, it is
also acknowledged that cognitive alterations, such as neural stimulation or
mind-altering drugs are an inherently invasive procedure, and that consent must
be informed and well judged. 

% TODO morals
