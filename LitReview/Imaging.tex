\section{Imaging and Image Processing}
At some point between a biological brain processing the life of an organism, and
a simulation of a brain attempting to continue computation, the brain must be
"copied" over. This can be done through several different imaging techniques.

\begin{figure}[h]
    \centering
    \includegraphics{figures/graphs/scaleexample.png}
    \DoubleCaption{scaleexample}
    {temp}
    \label{scaleexample}
\end{figure}
\vspace{1ex}

\subsection{Current Methods of Scanning the brain}

\subsubsection*{MRI Scanning}

- Common procedure
- Enough detail for general structure, but provides little to no insight to fine
details such as placement of dendritic!Typo?! spines or the exact size or placement of
synapses between neurons.
- As the patient is typically alive during a scan, there will be significant
measurement drift as the current is still moving around the brain as normal.

\subsubsection*{XRay}

- Not naturally useful, have to inject a substrate into the brain for this to
create an image of any meaningful contrast ?!?!CITE

\subsubsection*{Electron Microscope Scanning}

- Very precise
- Can take a very long time
- Need well "frozen" slices of the brain for this to be of any use or you'll
have major measurement drift as the brain decays.

\subsubsection*{New Methods}

- Some of the more experimental stuff here.

\subsection[Error induced through noise]{Examination of Error induced through the imaging process}

Prediction of future brain activity through simulation requires an accurate and
detailed connectome of a brain, with synapses and neurons correctly located in
space.\autocite{bostrom_whole_2008} The accuracy of such a model depends on the
resolution of the imaging method used to create it. The error resulting from
such a imaging method is the measurement error. Depending on imaging procedure, brain matter may shift in composition during the course of the scan, which is the cause of measurement drift, itself a form of measurement error.

\setlength{\tabcolsep}{3.3ex}
\renewcommand{\arraystretch}{1.1}
\begin{table}[h!]
    \centering
    \addtolength{\leftskip} {-0.5cm}
    \addtolength{\rightskip}{-0.5cm}
    \begin{tabular}{@{}lllll@{}}
        Method              & Environment & Resolution ($\mu m$ )                & Time       &
        Error (approx. \%)                                                                \\
        \hline
        MRI                 & In Vivo     & 6                  & 30 minutes &
        95                                                                          \\
        MRI microscopy      & In Vitro    & 3                   & -          &
        85                                                                          \\
        XRay microscopy     & In Vitro    & 0.03                       & -          &
        30                                                                          \\
        Electron microscopy & In Vitro    & \textasciitilde 0.03-0.001 & > 3 months &
        <1                                                                          \\
        Theoretical Ideal   & Either      & <0.005                       & <500s
        & <1
        \\
        \hline
    \end{tabular}
    \caption[Table comparing brain imaging methods.]{Table comparing brain
    imaging methods. Time taken and error is extrapolated over the whole human
    brain. Error is approximated from size of dendritic spines. Data primarily
    sourced from \autocite{bostrom_whole_2008, kaynig_large-scale_2015}}
    \label{imagemethodcomparison1}
\end{table}
\setlength{\tabcolsep}{1ex}

\subsubsection[Error induced through low resolution]{Effect of Low Resolution} 
(this is easy just print something out
and scan it badly. Explain what kind of error in incurred.)