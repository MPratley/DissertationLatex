\chapter{Design, Development and Methodology}

In order to better understand the fundamentals of computational neuroscience, I
am taking a "from scratch" approach to building this project. While this
increases the time required to run simulation experiments, this approach allows
me to build confidence in my understanding of the maths and models that are
commonly used in the wider computational neuroscience community.

\section{Simulation Requirements}

\begin{itemize}
    \item \textbf{Spiking neural network is the target emulation layer}
    Justification should cite imaging requirements and state that this is an
    appropriate level of abstraction to answer the question of this
    dissertation. Refer to the table of emulation layers in lit review.
    \item \textbf{Requirement 2.} Reasoning for requirement two.
\end{itemize}

\section{Modelling mathematics}
"In order to program this thing I must first enumerate the differential equations
that describe the functioning of Neurons in this spiking neural network."
\subsection{Leaky Integrate and Fire}

TEMP:
\begin{figure*}[h]
    \centering
    \begin{equation}\label{eq:LIF_TC}
        T_C
    \end{equation}
    \begin{equation}\label{eq:LIF_RC}
        \frac{d V}{d t} = -\frac{V}{\tau_L} + \frac{I(t)}{C},
    \end{equation}
    \begin{equation}\label{eq:integ_LIF_RC_VL}
        V(t)= \frac{1}{C} \int_{0}^{t} e^{-\frac{(t-s)}{\tau_L}} I(s) ds
    \end{equation}
    $C$ is the membrane capacity, $g_L$ is the leak conductance and $V_L$ is the leak reversal potential.
    % \caption{Formulae for a Leaky Integrate and Fire Neuron (pre-threshold)}
    \label{LIFequation}
\end{figure*}

\subsection{Spike refractory periods}

\subsection{STDP}

\section{Transferring models to Python}

In order to flexibly mix
object oriented and function programming paradigms, I've chosen to develop these
models in the Python programming language. Python also has a healthy and active
ecosystem of libraries and development practices for scientific and statistical research.

Of particular note are the scipy libraries, \ldots

\subsection{Object oriented neurons}

\subsection{Synapses}

\subsection{Stochastic generation of networks}

\subsubsection{Grouping neurons}

\subsubsection{Function parameters instead of static parameters}

\subsection{Measuring spiking patterns across the network}

-Mark certain neurons as recording
-This can be done as a boolean flag on the neuron to tell it to keep a time
trace of its potential, or by inserting measurement objects into the simulation.
These measurement objects can take the form of 

\subsection{Comparing spiking patterns between two networks.}

Treating spikes as a binary on-off at a given point in time, it is possible to
calculate the hamming distance between two network spiking patters. 

