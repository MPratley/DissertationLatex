\chapter{Analysis and Evaluation}

\section{What do the results mean?}

\section{Are they successful? Are they expected?}

- Would have been good to investigate the effect of inhibitory synapses.

- When calculating the difference between spiking distributions, the area under the curve, the total accumulated potential, should have been the same to ensure the results were comparable with both Kulbach-Leibler and Earth Movers Distance. Due to the nature of the spiking pattern of the neurons, this was not possible to guarantee, and instead error bars were shown where appropriate. However it would have been preferable, time allowing, to collect multiple results with the same network sizes and mean weighting error, to calculate mean simulation error.

\section{Reflection on development methodology}

 - Reactive approach, the requirements of the software adapting as the results
 of experiments take shape.

 - This is substantially different my original plan of identifying hard
 requirements of the system and implementing. 

 - It's much easier to let loose requirements set the direction of travel and
 let experiments along the way determine the end result of the software.

 - This is partially due to an initial lack of knowledge around the problem
 domain. As I understood more about computational neuroscience, domains that
 seemed simple took on additional complexity, while previously intractable
 concepts became apparent.

- Another benefit of taking an experimental and incremental approach to the
development of the project is that tasks naturally break down into small and
meaningful chunks; each new piece of code does the correct amount of work to
solve a set purpose.

- The caveat to this approach is that it does not lend itself to a holistic
system design or consistent architecture. When defining and writing new
data experiments, the flexibility of decoupled components is ideal, but new code
tends to be purpose built and highly coupled. This would sometimes mean
re-writing code several times as previously finished code had new requirements. 

- In general, Conclusions, I’ve found that getting development methodology
“right” is both hard and sort of a misnomer. It depends on the product, the
team, their locations and the workload: what “right” means will change as your
product changes and you move from an experimental development cycle to one with
well formed requirements. One solid way of working throughout a project will
cause friction later down the line. 

A lot of time an effort both inside and out of the software development industry
is spent defining `Agile ways of working' \autocite{spolsky_you_2006}, however, from my experiences on this
project, it is better to be flexible and reject workflow rigidity. 

Flexibility in ways of working does have its drawbacks however; deadlines must
be set and adhered to throughout a project lest one finds themselves overworked
and unable to produce outcomes that matter externally. In this aspect, my
project would have benefited greatly from a larger sense of responsibility to
myself and the targets I set myself at the beginning of the year. By way of
example, I had originally planned that certain stages of my project could
overrun safely, while others must be completed on time. Sticking to this plan
was far harder than I originally anticipated, and in hindsight it would have been
wise to re-scope the project as soon as it was clearly the best course of action,
instead of carrying a metaphorical weight that it was clear couldn't reach the
finish line.

\section{Personal reflection of result success/lack thereof}

\section{Personal reflection of whole project experience}
 - Certain concepts were intractable for quite some time 
 - concepts that I considered understood made little sense when the time came to
 develop on them 

\section{Ethical considerations for brain simulation}


In "Taking superintelligence seriously: Superintelligence: Paths, dangers,
strategies", Nick Bostrom argues that \ldots
\autocite{bostrom_superintelligence_2014}

\subsection{Pain and suffering in brain simulation}

\autocite{dan_preventing_2019}

\subsection{The rights of a simulation}

\subsection{Kinder artificial intelligence}
probably could go in conclusion. Cite lesswrong dude.
% https://wiki.lesswrong.com/wiki/Friendly_artificial_intelligence

% Is a brain simulation alive? 
% If accurate enough, how separable is a human and a brain simulation from that human?
% Is it ethical to allow a brain simulation to feel pain? 