

\section{Transferring models to Python}

In order to flexibly mix object oriented and functional programming paradigms,
I've chosen to develop these models in the Python programming language. Python
also has a healthy and active ecosystem of libraries and development practices
for scientific and statistical research.

Of particular note are the SciPy libraries, \ldots



\subsection{Object oriented neurons}

In order to make the simulation as extensible as possible, I have chosen to
implement it in an object oriented manner, with inheritance present in places
where it makes sense to use it. [MAKE A QUICK UML DIAGRAM]

Advantage of oop is that all the pipework around cloning objects is handled
by the language with minimal effort, ideal for running similar simulation with
slight parameter differences.

\subsection{Synapses}

In order to model the passing of information through the simulation, there were
several approaches that I could take. In order to pick a solution, it is
necessary to map out the relationships between neurons in some sample topologies
and the features that need to be supported by the synapses in such topologies.

% \subsection{Grouping neurons}

% In order to 

\subsubsection{Stochastic generation of networks}

When creating large networks of neurons, it is simpler to organise them in
groups that are generated with random variables, and to stochastically link
neuron groups with synapse groups. 

\subsubsection{Function parameters instead of static parameters}

It is desirable to make the parameters that define the random distribution in a group
generation function be configurable at runtime, but increasing the number of
parameters in a function signature increases the mental overhead when a programmer is
using an API [CITE THIS]. 

However,
this can be simplified even further by passing a single PYTHON CALLABLE, which
in practice is a closure that encapsulates a function that produces a
distribution of the user's choice. 