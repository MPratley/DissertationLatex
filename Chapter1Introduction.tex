\chapter{Introduction}

The topic of creating artificial life has been subject to fervent debate for
hundreds of years, but at the advent of modern computing this conversation took
a turn towards the technological. In his 2005 book “The Singularity Is Near:
When Humans Transcend Biology” author Ray Kurzweil predicts not that we will
create a new species, but that “The Singularity will allow us to transcend [the]
limitations of our biological bodies and brains”
\parencite{kurzweil_singularity_2006}.
In short, we will be able to transfer our
consciousness to simulations of the human brain. The implication of this is
that, assuming inexhaustive power and processing to maintain the simulation,
humanity would be made immortal. However, there are a number of constraints
technologically that must be reckoned with before this could even be attempted
\parencite{bostrom_whole_2008}, alongside the moral quandaries that are ever
present when discussing the evolution and modification of what it means to be
human. 

When discussing brain emulation, we envisage the “uploading” of
consciousness, which in basic terms is the measurement of electrical activity
and neural connections of a brain, and in theory would retain every mental
process, keeping them intact. Using current known techniques, this
measurement process would introduce error in numerous ways; It is perhaps
possible that the resolution of the scanning equipment is not large enough to
correctly capture the distances between neurons, or that changes in the network
will occur while measurements are taken. It is therefore unclear what impact
this measurement error would have on a simulation of the human brain.

\section{Motivation}

This project aims to investigate the effect of measurement error on the accuracy
of a brain simulation with and without synaptic placticity, and to assess the
feasibility of predicting future brain activity, given the performance of
current imaging techniques.

The primary outcome of this project will be to provide Python tooling that is
capable of running simulations in a parallel manner across multiple threads,
that can be modified during runtime to model the effect of measurement error
when“uploading”a brain into a simulation. 

\section{Aims and Objectives}

\begin{itemize}
    \item \textbf{Analyse and compare three published neural simulation packages.}
          Developing an understanding of simulation packages and the models that
          they implement will help me inform my design decisions for my own
          implementation of this feature. This objective will have been
          completed when I have summarised and understood the aims and features
          of three existing packages.
    \item \textbf{Identify the experiments that should be performed to determine
    the relation between measurement error in data from imaging a brain, and the
    performance of a neural simulation of such data.} In order to prove that
    that measurement error has an effect on predicting future brain activity, I
    will define experiments that can be performed within the limits of this
    project. Each experiment must contribute either to the argument that
    measurement error has a major negative impact on the accuracy of a neural
    simulation, or that introducing plasticity into the simulation minimises
    this effect over time.
    \item \textbf{Identify the requirements and features that my simulation tooling should implement to be capable of performing the project experiments.}
    % \item \textbf{Identify the requirements of a routine within VERTEX that is
    %           capable of saving a snapshot of a running simulation.} I will need
    %       to gather data relating to the functionality and design of a
    %       simulation snapshot saving system. This will include the API
    %       presented to the user of the software, required data structures to
    %       store information, and choosing functionality which forms the
    %       minimal viable product. To achieve this, I must have collected
    %       enough research to make conclusive design and functionality
    %       decisions.
    \item \textbf{Identify the parameters in a brain simulation that can be
              modified during the simulation to emulate the measurement error of
              uploading the human brain.} Many of the parameters required in the
          creation of a simulation, such as the number of layers or the
          positions and links between neurons could be modified during a
          simulation using the tooling this project aims to develop. This
          objective will be complete when I have identified these
          parameters, and justified why they have been chosen over others.
    \item \textbf{Analyse the performance of simulations that diverge with
              different measurement errors from a starting simulation.} This
          final objective will assess whether the software designed and
          developed during the project can correctly demonstrate the effect
          of measurement error on brain simulation. 
\end{itemize}

\section{Experiments and Hypotheses}



\section{Document Structure}

\subsection*{Background and Literature Review}
This chapter aims to convey the general state of what is considered mainstraim
academic and industry opinion, and some of the more cutting edge approaches in
each of the domains that this project touches on.

\subsection*{Planning and Development}
This chapter will discuss the methodology by which project deliverables have
been developed, and explain the rationale of certain design decisions. In
particular, functionality of the neural network simulation will be
discussed, as will the simulation parameters.

\subsection*{Experiments and Results}
The rudimentary correctness of the Python simulation is verified here, and each
of the experiments that form the basis of the arguments made in this
dissertation are expanded on and their results are explained. 

\subsection*{Analysis and Evaluation}
An analysis and evaluation will review the results in the previous chapter and
show how the results reflect on the overall project aims. The results of all the experiments are reviewed holistically and compared with
expected findings. 
\subsection*{Conclusion}
A summary of the project, and whether the aim and objectives have been met. The
overall sucess of the project is evaluated and future work that could be
completed in another project expanded on.